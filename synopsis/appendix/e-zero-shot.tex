\chapter{Zero-Shot Classification}
\label{appendix:zero-shot}

Zero-shot classification is a machine learning technique that allows a model to assign labels to input text without having seen labeled examples for those specific classes. It relies on models trained on general-purpose tasks such as Natural Language Inference (NLI).

\section*{Why Zero-Shot?}
In this project, we did not have human-labeled genres for the books. Instead, we defined a set of target categories (e.g., \textit{Fantasy}, \textit{Science Fiction}, \textit{Historical}) and used a pretrained NLI model to decide whether a description "entails" each candidate label.

\section*{NLI-Based Classification}
The model used was \texttt{facebook/bart-large-mnli}, a transformer trained on premise-hypothesis relationships.

Given:
\begin{itemize}
  \item Premise: the book description
  \item Hypothesis: "This book is \textit{[label]}"
\end{itemize}
The model evaluates whether the hypothesis is entailed by the premise. A high entailment probability means the label is likely correct.

\section*{Multi-Label Inference}
Zero-shot classification supports multi-label predictions. A single description may be associated with multiple categories. Only predictions above a threshold (e.g., 0.4) were retained.

\section*{Advantages}
\begin{itemize}
  \item No training required on project-specific data
  \item Easily extendable to new label sets
  \item Strong generalization using a well-trained NLI base
\end{itemize}

\section*{Limitations}
\begin{itemize}
  \item Performance depends on hypothesis phrasing (e.g., "This book is \textit{fantasy}" vs "This story involves \textit{fantasy}")
  \item May confuse overlapping or abstract categories
  \item Computationally slower than simple keyword matching
\end{itemize}

Zero-shot classification enabled this project to assign interpretable categories to books without manual labeling, supporting both UI filtering and statistical analysis.
