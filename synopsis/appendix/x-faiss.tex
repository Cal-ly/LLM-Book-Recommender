\chapter{Similarity Search with FAISS}
\label{appendix:faiss}

Similarity search is the task of finding items most similar to a given query in a high-dimensional vector space. This project uses FAISS (Facebook AI Similarity Search) to perform fast, exact nearest-neighbor lookups over book embeddings.

\section*{Motivation}
Books and queries are embedded into the same semantic vector space. To recommend relevant books, the system retrieves the closest vectors (books) to a given query vector.

\section*{What is FAISS?}
FAISS is a C++/Python library developed by Meta for fast search across large collections of dense vectors. It supports both exact and approximate search methods.

\section*{Indexing Method Used}
This system uses the exact search index:
\begin{itemize}
  \item \texttt{IndexFlatL2} — Computes L2 (Euclidean) distance between the query and all book vectors
  \item Suitable for small to medium datasets (\textasciitilde5,000 vectors)
  \item No compression or quantization
\end{itemize}

\section*{Search Workflow}
\begin{enumerate}
  \item Book metadata is converted to \texttt{search\_text} and embedded
  \item Vectors are added to a FAISS index
  \item At query time, the user input is embedded
  \item FAISS returns top-$k$ nearest neighbors
  \item Metadata is retrieved from the indexed CSV
\end{enumerate}

\section*{Distance Metric}
FAISS computes squared L2 distance:
\[ \text{dist}(\vec{q}, \vec{b}_i) = \| \vec{q} - \vec{b}_i \|_2^2 \]

Lower values indicate higher semantic similarity.

\section*{Benefits}
\begin{itemize}
  \item Extremely fast for exact matching on CPU
  \item Integrates easily with NumPy and PyTorch
  \item Fully offline and open source
\end{itemize}

\section*{Alternatives}
For larger datasets, approximate methods like HNSW, IVF, or PQ (quantization) may be more scalable but require tuning.

FAISS was ideal for this project due to its simplicity and performance at the given scale.