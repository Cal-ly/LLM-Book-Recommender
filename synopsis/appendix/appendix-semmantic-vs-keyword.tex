\chapter{Semantic vs. Keyword Search}
\label{appendix:semantic-vs-keyword}

Traditional keyword search retrieves documents containing exact word matches. Semantic search, by contrast, compares the meaning of queries and texts using vector representations. This project uses semantic search to match user queries with relevant books, regardless of vocabulary overlap.

\section*{Keyword Search}
Keyword-based systems look for exact or partial term matches. For example:
\begin{quote}
\texttt{Query: "wizard school"} \rightarrow matches: "The Wizard's Handbook", \textit{but not} "A Magical Academy in the North"
\end{quote}

\section*{Semantic Search}
Semantic search embeds both the query and the dataset items into a shared vector space. Results are ranked by vector similarity rather than literal term overlap.

\textbf{Example:}
\begin{quote}
\texttt{Query: "wizard school"} \rightarrow may match: "Hogwarts: A History", "Spellbound Scholars", or "Magic and Adolescence"
\end{quote}

These may not contain the query words but share conceptual meaning.

\section*{Advantages of Semantic Search}
\begin{itemize}
  \item Captures meaning beyond surface-level keywords
  \item Handles synonyms, paraphrasing, and reworded concepts
  \item More resilient to spelling errors and phrasing variation
\end{itemize}

\section*{When Keyword Search is Better}
\begin{itemize}
  \item When searching for exact titles or phrases
  \item When very high precision is required
  \item For proper nouns and rare names (e.g., author surnames)
\end{itemize}

\section*{Hybrid Approaches}
Some systems combine keyword and semantic methods: keywords may be used to pre-filter candidates, followed by semantic ranking. This balances recall and precision.

For this project, semantic search enables free-form exploration, while metadata fields (like author and tags) provide structure for post-filtering.
