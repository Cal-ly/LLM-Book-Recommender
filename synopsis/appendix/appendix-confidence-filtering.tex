\chapter{Confidence Filtering}
\label{appendix:confidence-filtering}

Confidence filtering is the process of retaining only high-certainty predictions from a model. In this project, it is used to ensure that category labels assigned to books are reliable.

\section*{Motivation}
Zero-shot classification produces a confidence score for each category. However, not all scores are meaningful — low-confidence predictions can lead to noisy or irrelevant labels.

\section*{Metrics Used}
Three metrics were computed for each book's classification output:

\begin{itemize}
  \item \textbf{max\_score} — the highest confidence score among all categories
  \item \textbf{filtered\_avg\_score} — average confidence score for predictions above a threshold (e.g., 0.2)
  \item \textbf{score\_std} — standard deviation of all confidence scores
\end{itemize}

\section*{Filtering Strategy}
To retain high-quality entries for indexing and UI use, books were required to meet all of the following:

\begin{itemize}
  \item \texttt{description\_length} $\geq$ 200 characters
  \item \texttt{filtered\_avg\_score} $\geq$ 0.2
  \item \texttt{max\_score} $\geq$ 0.4
  \item At least one predicted category
\end{itemize}

\section*{Why Multiple Metrics?}
\begin{itemize}
  \item \textbf{max\_score} ensures at least one strong category signal
  \item \textbf{filtered\_avg\_score} avoids noisy averages dominated by low scores
  \item \textbf{score\_std} (analyzed but not thresholded) helps detect ambiguous predictions
\end{itemize}

\section*{Impact}
These thresholds reduced the dataset from 6,572 to 5,160 entries, each with well-supported category labels and semantically rich descriptions. This filtering greatly improved the quality of recommendations.

Confidence filtering balances recall and precision, ensuring that the final indexed dataset is useful and trustworthy.
