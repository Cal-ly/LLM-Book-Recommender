\chapter{Methodology and Structure}
\label{chapter:methodology}

This project follows a research-driven prototype methodology aimed at exploring the feasibility of using local language models for semantic book recommendations. The goal is not to build a commercial product, but to test how well local ML tools can perform end-to-end recommendation without internet access.

\section{Research Approach}
\label{sec:research-approach}

The methodology combines literature review, modular implementation, and empirical testing. Each decision supports the research question and sub-questions (see Section~\ref{sec:problem-definition}).

\begin{itemize}
    \item \textbf{Literature Review:} Key areas included sentence embeddings, content-based recommenders, zero-shot classification, and privacy-aware ML. Emphasis was placed on open-source tools and local-first design \cite{handson-ml}.
    
    \item \textbf{Implementation:}
    \begin{itemize}
        \item \texttt{Python} + \texttt{pandas} for scripting and orchestration.
        \item \texttt{sentence-transformers} (\texttt{MiniLM-L6-v2}) for text embeddings.
        \item \texttt{facebook/bart-large-mnli} for zero-shot category inference.
        \item \texttt{FAISS} for semantic vector indexing.
        \item \texttt{Streamlit} for an interactive frontend.
    \end{itemize}

    \item \textbf{Data Processing:} Categories were refined using confidence scores, fallbacks, and correlation-based filtering.
    
    \item \textbf{Testing:} A variety of query types were tested with and without filters to evaluate relevance and robustness.
\end{itemize}

\section{Evaluation Criteria}
\label{sec:evaluation-criteria}

No labeled test set was available, so evaluation focused on:

\begin{itemize}
    \item \textbf{Semantic relevance} — how well results match query intent.
    \item \textbf{Responsiveness} — ability to respond quickly on CPU-only hardware.
    \item \textbf{Filtering utility} — impact of genre/rating filters.
    \item \textbf{Offline execution} — system runs without network access.
\end{itemize}

These metrics align with sub-questions~\ref{itm:subq-embedding}, \ref{itm:subq-similarity}, and \ref{itm:subq-classification}.

\section{Structure of the Synopsis}
\label{sec:structure-synopsis}

\subsection{Main Chapters}
The main chapters cover the core components of the system, with each addressing a specific aspect of the methodology.

\subsection{Appendices}
Supporting materials include example code, glossary terms, and supporting formulas.

\section{Application Architecture}
\label{sec:application-architecture}

The overall system is modular and layered:

\begin{tikzpicture}[node distance=1.5cm]
    \node (desc) [block] {Input: Book Metadata};
    \node (augment) [block, below of=desc] {Augment + Clean Description};
    \node (classify) [block, below of=augment] {Category Inference (ZSC + Fallback)};
    \node (filtering) [block, below of=classify] {Score-Based Filtering};

    \node (embed) [block, below of=filtering] {MiniLM Embedding};
    \node (index) [block, below of=embed] {FAISS Index};

    \node (query) [block, right of=desc, xshift=6cm] {User Query};
    \node (qembed) [block, below of=query] {MiniLM Embedding (Query)};
    \node (search) [block, below of=qembed] {FAISS Similarity Search};
    \node (refine) [block, below of=search] {Apply UI Filters (Tags, Rating)};
    \node (output) [block, below of=refine] {Display Top-K Recommendations};

    \draw [arrow] (desc) -- (augment);
    \draw [arrow] (augment) -- (classify);
    \draw [arrow] (classify) -- (filtering);
    \draw [arrow] (filtering) -- (embed);
    \draw [arrow] (embed) -- (index);

    \draw [arrow] (query) -- (qembed);
    \draw [arrow] (qembed) -- (search);
    \draw [arrow] (index) -- (search);
    \draw [arrow] (search) -- (refine);
    \draw [arrow] (refine) -- (output);
\end{tikzpicture}
