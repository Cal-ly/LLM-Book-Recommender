\chapter{Introduction}
\label{chapter:introduction}

\section{Motivation}
\label{sec:motivation}

The purpose of this project is to explore the feasibility of running local language models in privacy-preserving environments. To this end, a book recommendation system has been selected as the practical foundation. The recommender leverages Natural Language Processing (NLP) techniques to analyze both book descriptions and free-form user queries, enabling content-based recommendations.

In contrast to most cloud-based recommendation systems, this project demonstrates a fully local setup where all data processing and inference occurs on the user's own machine. This ensures that no user data is transmitted or stored externally, providing strong privacy guarantees.

The goal is to evaluate whether smaller transformer-based models, specifically sentence transformers like MiniLM, are capable of delivering high-quality recommendations in an offline context. Techniques such as semantic embedding, vector similarity search, and metadata filtering are investigated to improve recommendation quality. In doing so, the project explores a broader question of how accessible and effective local AI systems can be for individual users.

\section{Problem Definition}
\label{sec:problem-definition}

The project is centered around the following research question:

\begin{quote}
\textit{How can a local ML model be used to recommend books based on natural language descriptions, relying only on locally running models?}
\end{quote}
\label{itm:main-question}

This leads to three guiding sub-questions:

\begin{enumerate}
    \item \label{itm:subq-embedding} \textit{What techniques exist for embedding text into meaningful vectors?}
    \item \label{itm:subq-similarity} \textit{How can vector similarity be used for finding similar books?}
    \item \label{itm:subq-limitations} \textit{What are the limitations of a local, content-only recommender?}
\end{enumerate}

These sub-questions form the conceptual framework for the research and analysis presented in the following chapters.

\section{Planning}
\label{sec:planning}

The project was conducted over five weeks and divided into the following phases:

\begin{itemize}
    \item \textbf{Week 1 –} Literature review and background research on semantic similarity, embedding models, and recommender systems.
    \item \textbf{Week 2 –} Data exploration and cleaning, including handling of missing values and metadata normalization.
    \item \textbf{Week 3 –} Embedding and indexing: generating dense sentence embeddings and setting up FAISS for similarity search.
    \item \textbf{Week 4 –} Development of a prototype user interface and live experimentation with queries and filters.
    \item \textbf{Week 5 –} Final testing, evaluation, and writing of the synopsis.
\end{itemize}

