\chapter{User Interface}
\label{chapter:interface}

While not explicitly part of ML, the design and logic of the user interface (UI) enables interactive querying and filtering of book recommendations.
The frontend was built using \texttt{Streamlit}, a lightweight Python-based web framework suitable for rapid prototyping and local execution.

\section{Design Principles}
\label{sec:ui-design}

The UI was designed with simplicity, usability, and privacy in mind. All processing, including embedding, similarity search, and filtering, takes place locally. No user data is sent over the internet.

Key goals included:

\begin{itemize}
    \item Provide an intuitive way for users to search by concept or theme, not keywords.
    \item Allow flexible filtering based on metadata such as rating or genre.
    \item Ensure responsiveness without requiring server-side infrastructure.
\end{itemize}

---

\section{Query Workflow}
\label{sec:ui-query-workflow}

The interface presents a single input box where the user can enter a free-form natural language description, such as:

\begin{quote}
    \texttt{“A dystopian society controlled by AI”}
\end{quote}

Once submitted:

\begin{enumerate}
    \item The query is embedded using the same MiniLM model used for the book descriptions.
    \item The query vector is searched against the FAISS index to return the top 50 results.
    \item Results are filtered and sorted based on user preferences.
\end{enumerate}

---

\section{Filtering and Sorting Logic}
\label{sec:ui-filtering}

The UI provides several mechanisms for controlling output:

\begin{itemize}
    \item \textbf{Minimum rating filter:} A slider allows users to set a threshold (e.g., $\geq 4.0$).
    \item \textbf{Genre/category filter:} A case-insensitive text match on the \texttt{categories} field.
    \item \textbf{Sorting:} Results can be sorted by average rating in descending or ascending order.
\end{itemize}

These filters are applied after similarity search, allowing semantic results to be further refined by user criteria.

---

\section{Result Display}
\label{sec:ui-display}

Each result includes:

\begin{itemize}
    \item \textbf{Book cover image} (fetched from OpenLibrary using ISBN, with fallback).
    \item \textbf{Title and author(s)} in formatted text.
    \item \textbf{Average rating} using a star display.
    \item \textbf{Excerpt of the book description} (up to 500 characters).
\end{itemize}

Results are displayed in a paginated layout (6 books per page), with navigation controls provided via a page selector widget.

---

\section{Privacy and Offline Execution}
\label{sec:ui-privacy}

All user inputs and results are processed locally. The system:

\begin{itemize}
    \item Does not collect or transmit any user data.
    \item Does not require login, cookies, or tracking.
    \item Functions without an internet connection (except for optional book cover lookups).
\end{itemize}

This design supports the overarching motivation for the project: to demonstrate the viability of running compact, privacy-respecting ML models on local hardware.

---

\section{Summary}
\label{sec:ui-summary}

The user interface enables real-time, offline interaction with the recommender system. Streamlit provides a simple and effective platform for building and deploying such applications, and the combination of semantic search with filtering yields a user experience that is both flexible and secure.
