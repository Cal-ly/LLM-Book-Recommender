\begin{frame}{Critical Analysis \& Future Directions}

\begin{columns}[T]
  \begin{column}{0.50\textwidth}
    \textbf{Implementation Strengths:}
    \begin{itemize}
      \item \highlight{Privacy-preserving} by design
      \item \highlight{Semantic understanding} beyond keyword matching
      \item \highlight{Lightweight} - runs on consumer hardware
      \item \highlight{Modular architecture} for easy extension
    \end{itemize}

    \vspace{0.3cm}
    \textbf{Current Limitations:}
    \begin{itemize}
      \item \highlight{No personalization} - stateless by design
      \item \highlight{Dataset scope} - 5,160 books vs. commercial scale
      \item \highlight{Cold start problem} for new books
      \item \highlight{No feedback learning} - static recommendations
    \end{itemize}

    \vspace{0.3cm}
    \textbf{Edge AI:}
      \begin{itemize}
        \item Edge AI is popular term for local-first ML
        \item Growing trend in mobile, home, IoT, privacy applications
        \item The project fits this paradigm perfectly
      \end{itemize}
  \end{column}

  \begin{column}{0.45\textwidth}
    \textbf{Future Research Directions:}
    \begin{itemize}
      \item \highlight{Hybrid approach:} Combine content-based with collaborative filtering
      \item \highlight{Better embeddings:} Experiment with domain-specific models
      \item \highlight{Privacy-preserving personalization:} Local user preference learning
      \item \highlight{Multi-modal features:} Include cover images, genre embeddings
    \end{itemize}

    \begin{center}
    \begin{beamercolorbox}[sep=8pt,center,rounded=true]{block body}
    \scriptsize \textit{"Demonstrates that local-first ML - or using a more popular term - edge AI, can provide meaningful semantic recommendations without compromising user privacy or requiring cloud infrastructure."}
    \end{beamercolorbox}
    \end{center}
  \end{column}
\end{columns}

\note{
  \begin{columns}[T]
    \begin{column}{0.49\textwidth}
      [TIMING: 1 minute - HONEST SELF-ASSESSMENT:]
      \begin{itemize}
        \item "This is proof-of-concept, not production system"
        \item "Goal was to demonstrate feasibility, not perfection - Garbage In, Garbage Out"
      \end{itemize}
      
      \vspace{0.1cm}
      [LIMITATIONS:]
      \begin{itemize}
        \item "No personalization - everyone gets same results for same query"
        \item "Dataset small compared to Amazon's millions of books"
        \item "New books need manual addition and embedding - batch trained"
      \end{itemize}
      
      \vspace{0.1cm}
      [FUTURE DIRECTIONS:]
      \begin{itemize}
        \item "Hybrid: Add collaborative filtering while preserving privacy and independence"
        \item "Better embeddings: Fine-tune MiniLM on book-specific corpus"
        \item "Local personalization: Session-based learning without tracking"
        \item "Multi-modal: Computer vision on book covers for genre signals"
      \end{itemize}
    \end{column}
    
    \begin{column}{0.49\textwidth}

      [POTENTIAL QUESTIONS:]
      \begin{itemize}
        \item \textit{"How would you add personalization?"} → "Local preference vectors, federated learning, session-based adaptation"
        \item \textit{"Could this work for other domains?"} → "Yes - movies, music, research papers, any content with descriptions"
        \item \textit{"What's the biggest technical limitation?"} → "Embedding quality for domain-specific queries"
        \item \textit{"How evaluate recommendation quality?"} → "User studies, semantic similarity benchmarks, domain expert evaluation"
      \end{itemize}
      
      \vspace{0.2cm}
      [TRANSITION:]
      \begin{itemize}
        \item "Let me conclude by answering the original research question..."
      \end{itemize}
    \end{column}
  \end{columns}
}

\end{frame}